%%
%% Tikz styles used for diagrams
%%

\tikzstyle{every node}+=[font=\footnotesize \sffamily]

\tikzstyle{dropshadow} = [blur shadow={shadow blur steps=5,shadow xshift=.0ex,
                                       shadow yshift=-0.3ex,opacity=0.9,
                                       shadow blur radius=0.5ex}]

\tikzstyle{compound} = [rectangle, draw, text centered,
                         rounded corners,
                         top color=white,
                         bottom color=black!5,
                         dropshadow,
                         draw=black!10]

\tikzstyle{component} = [compound, draw=black!70]

\tikzstyle{compoundlabel} = [below right,
                             text opacity=0.6, inner sep=0, outer sep=0.7ex]

\tikzstyle{capability} = [circle, draw, fill, thick,
                          inner color=white,
                          outer color=blue!30,
                          inner sep=1pt,
                          minimum size=2.8ex,
                          fill opacity=1.0, dropshadow,
                          anchor=mid]

\tikzstyle{capslot} = [minimum size=2.5ex, inner sep=0ex, outer sep=0ex,
                       fill=white, draw=black!70]

\tikzstyle{kernelobj} = [draw, tape, tape bend top=none,
                         draw=black!70, align=center, fill=white, dropshadow]

\tikzstyle{arrow} = [draw=black!80, ->, >= stealth, sloped, above]

\tikzstyle{userland} = [fill=green]

\tikzstyle{kernel} = [fill=yellow]

\tikzstyle{thread} = [font=\normalsize]

\tikzstyle{nodecompound} = [rectangle, draw, align=center,
                            minimum height=2em, minimum width=2em,
                            inner sep=1.5ex, outer sep=1ex,
                            rounded corners,
                            top color=white,
                            bottom color=black!20,
                            dropshadow,
                            draw=black!10, thick]

\tikzstyle{ownership} = [draw, single arrow, draw=black!10, minimum height=4ex,
                         bottom color=white, top color=red!70,
                         single arrow head extend=1ex]

\tikzstyle{every picture}+=[remember picture]

\newcommand\kernelred{black!20!red!100}

\tikzstyle{dataspace} = [tape, draw, tape bend top=none, fill=white,
                         dropshadow, minimum height=6ex, align=center,
                         tape bend height=0.5ex, fill opacity=0.95]

% dotted line that connects dataspaces mapped in different address spaces
\tikzstyle{dataspacemapped} = [draw, draw, dotted]

% style representing a hardware component
\tikzstyle{hardware} = [draw=black!50, dropshadow, align=center,
                        path fading=flow fade, thin,
                        top color=black!5, bottom color=black!20]

%%
%% Styles for flow charts
%%
\tikzstyle{flowborder} = [draw=black!50]

\tikzstyle{flowbox} = [flowborder, fill=white, dropshadow,
                       rounded corners=0.5ex,
                       fill=yellow!20,
                       path fading=flow fade]

\tikzstyle{flowdecision} = [flowborder, diamond, fill=blue!20,
                           text badly centered, align=center,
                           inner sep=0.5ex,
                           dropshadow, path fading=flow fade]

%%
%% Styles for timeline diagrams
%%

\tikzstyle{activetimeline} = [draw,
                              blur shadow={shadow blur steps=5},
                              shadow yshift=-0.3ex, shadow xshift=0em,
                              rounded corners=0.4ex,
                              color=black!30,
                              path fading=timeline fade right,
                              left color=yellow!90!white!50,
                              right color=yellow!100!white!50,
                              minimum width=2.5ex]

\tikzstyle{timelinemessage} = [draw=black!80, ->, >= stealth, sloped,
                               shorten >= 0.8ex, shorten <= 0.8ex]

\tikzstyle{timelinegroup} = [draw=black!50, dashed,
                             rounded corners, fill=black, fill opacity=0.05]


%%
%% Styles for UML class diagrams
%%

\tikzstyle{umlclass}  = [draw, fill=white, dropshadow, align=left]
\tikzstyle{umlclass2} = [umlclass, rectangle split, rectangle split parts=2]
\tikzstyle{umlclass3} = [umlclass, rectangle split, rectangle split parts=3]

\tikzstyle{umlinfo} = [umlclass,
                       chamfered rectangle,
                       chamfered rectangle corners=north east,
                       text opacity=0.5, draw opacity=0.5, align=center]

\tikzstyle{umlinfoline} = [draw, draw opacity=0.5, dotted, thick]
\tikzstyle{umlinherit} = [draw, ->, >= open triangle 60]
\tikzstyle{umlrelated} = [draw, above]
\tikzstyle{umloperate} = [arrow, dashed, above]


%%
%% Functions for constructing Tikz nodes
%%

%%
% Create capability space node
%
% argument 1: node name, also used as prefix for the individual slots
% argument 2: maximum index
% argument 3: node arguments
%
% Uses tikz style "capslot"
%
\newcommand\capspacenode[3]{
	\node[#3] (#1) {
		\begin{tikzpicture}
			\begin{scope}[start chain,node distance=0,rounded corners=0]
				\foreach \i in {0,1,...,#2} {
					\node [on chain,capslot,draw] (#1\i) {};
					\path (#1\i) node[yshift=0.5ex, xshift=-0.5ex, inner sep=0.5ex,
					                  text=black!60] {\tiny \i};
				}
				\node [on chain,capslot, fill=none, draw=none, outer sep=0.5ex] {$\ldots$};
			\end{scope}
		\end{tikzpicture}
	};
}


%%
% Create RPC object node
%
% argument 1: node name, also used as prefix for the individual slots
% argument 2: node arguments
% argument 3: capability name
% argument 4: node content
% argument 5: capability node arguments
%
% The capability node will be named #1cap.
% The object node will be named #1obj.
%
\newcommand\rpcobjectnodebase[5] {
	\node[#2] (#1) {
		\begin{tikzpicture}
			\node [draw, left color=blue!20, right color=white, minimum size=0pt,
			       thick, inner sep=1ex, rounded corners=0.5ex, dropshadow]
			       (#1obj) {#4};
			\node [outer sep=0pt, capability, #5]
			       (#1cap) {#3};
			\path [draw, dropshadow, very thick] (#1cap) -- node (#1link) {} (#1obj);
		\end{tikzpicture}
	};
}


%%
% Create RPC object node with the capability to the right of the object
%
% The four arguments are the same as for 'rpcobjectnodebase'.
%
\newcommand\rpcobjectrnode[4] {
	\rpcobjectnodebase{#1}{#2}{#3}{#4}{right=1.5ex of #1obj} }


%%
% Create RPC object node with the capability to the left of the object
%
% The four arguments are the same as for 'rpcobjectnodebase'.
%
\newcommand\rpcobjectlnode[4] {
	\rpcobjectnodebase{#1}{#2}{#3}{#4}{left=1.5ex of #1obj} }


%%
% Create compounding node
%
% argument 1:  node name
% argument 2:  node style
% argument 3:  node content
%
\newcommand\compoundnode[3] {
	\node[#2] (#1) {
		\begin{tikzpicture}[xshift=0, yshift=0, outer sep=0, inner sep=0,
		                    rounded corners=0]
			#3
		\end{tikzpicture}
	};
}


%%
% Create labeled compounding box node
%
% argument 1:  node name
% argument 2:  node style
% argument 3:  label to appear at the north-west corner
% argument 4:  content of the box
%
\newcommand\labeledcompoundnode[4] {
	\compoundnode{#1}{inner sep=3ex, #2}{#4}
	\path (#1.north west) node[compoundlabel] {#3};
}


%%
% Create kernel-user boundary
%
% argument 1:  nodes that are contained in the kernel,
%              specified as tikz style, e.g., 'fit=(node1) (node2)...'
%
\newcommand\kernelboundary[1] {
	\node [inner sep=4ex, #1] (kernelboundary) {};
	\draw [dashed, very thick, color=\kernelred] (kernelboundary.north west)
	      -- (kernelboundary.north east)
	      node[below left] {kernel};
}


%%
% Create node for an upward pointing thread
%
% argument 1:  node name
% argument 2:  node style
%
\newcommand\upwardthreadnode[2] {
	\node [thread, #2] (#1) {$\uplsquigarrow$}; }


%%
% Create node for an downward pointing thread
%
% argument 1:  node name
% argument 2:  node style
%
\newcommand\downwardthreadnode[2] {
	\node [thread, #2] (#1) {$\downrsquigarrow$}; }


%%
% Create entrypoint node
%
% argument 1:  node name
% argument 2:  node style
% argument 3:  component where the entrypoint resides
%
% The command creates sub nodes for the semicircle called #1ep and the
% thread called #1thread.
%
\newcommand\entrypointnode[3]{
	\node [above, inner sep=0, outer sep=0, #2] at (#3.south) (#1) {
		\begin{tikzpicture}

			\node [draw=black, draw opacity=0.4, ball color=blue, fill opacity=0.2,
			       rounded corners=0, shape=semicircle,
			       inner sep=1.3ex, outer sep=0, above]
			(#1ep) {};

			\path (#1ep.arc end) node[compoundlabel, right, yshift=1ex] {EP};

			\path (#1ep.arc start) [thread]
				node [left, xshift=-0.3ex, yshift=1ex] (#1thread) {$\uplsquigarrow$};

		\end{tikzpicture}
	};
}


%%
% Create path that illustrates a capability delegation
%
\newcommand\delegationpath{
	\path [color=black!30!blue!30, text=black,
	       decoration={markings,
	                   mark=between positions 0.03 and 1 step 1ex with {\arrow{latex}}},
	       postaction={decorate}]}


\newcommand\capability[1]{
	\begin{tikzpicture}[baseline=-0.6ex]
	\node [capability] {#1};
	\end{tikzpicture}
}


%%
%% Timeline diagrams
%%

\newcounter{numtimelines}


%%
% Create new vertical timeline
%
% argument: timeline name
%
\newcommand{\newtimeline}[1] {

	\stepcounter{numtimelines}

	\path (timelineanchor) node (#1)     {};
	\path (timelineanchor) node (last#1) {};

	\pgfkeyssetvalue{/timeline\thenumtimelines/name}{#1}

	\pgfkeyssetvalue{/timeline #1/state}{idle}
}


%%
% Draw vertical state representation of the specified timeline (internal)
%
% argument: timeline name
%
\newcommand{\finishtimeline}[1] {

	\def\state{\pgfkeysvalueof{/timeline #1/state}}

	\ifthenelse{\equal{\state}{idle}}{

		\draw[dashed, thick, draw opacity=0.1] (last#1) -- (#1);
	}{}

	\ifthenelse{\equal{\state}{active}}{

		\path (last#1)+(0,-0.1) node (top)    {};
		\path     (#1)+(0, 0.1) node (bottom) {};

		\path (last#1) node[inner sep=0pt, fit=(top) (bottom), activetimeline] {};
	}{}

	\path (#1) node (last#1) {};
}


%%
% Timeline environment
%
% The environment can only be used within a tikzpicture environment
%
\newenvironment{timelinediagram} {

	% reset timeline counter
	\setcounter{numtimelines}{0}

	% reset anchor used for next call of 'newtimeline'
	\path (0,0) node (timelineanchor) {};

	\newcommand{\timestepvector}{0,-1}

	%%
	% Add time step
	%
	\newcommand{\timestep} {

		\foreach \i in {1,...,\thenumtimelines}{

			\def\name{\pgfkeysvalueof{/timeline\i/name}}

			% move timeline forward in time
			\path (\name)+(\timestepvector) node (\name) {};
		}
	}

	%%
	% Activate timeline
	%
	\newcommand{\activate}[1] {
		\finishtimeline{##1}
		\pgfkeyssetvalue{/timeline ##1/state}{active}
	}

	%%
	% Deactivate timeline
	%
	\newcommand{\deactivate}[1] {
		\finishtimeline{##1}
		\pgfkeyssetvalue{/timeline ##1/state}{idle}
	}

	%%
	% Create transition from one timeline to another
	%
	\newcommand{\transition}[3] {
		%
		% The named 'transitionlabel' node can be used to attach further
		% information to the message. The unnamed second node is designated
		% for labeling simple transitions.
		%
		\draw[timelinemessage] (##1) -- node (transitionlabel) {} node[above] {##3} (##2);
		\deactivate{##1}
		\activate{##2}
	}

} {
	% finish all timelines
	\foreach \i in {1,...,\thenumtimelines}{
		\def\name{\pgfkeysvalueof{/timeline\i/name}}
		\finishtimeline{\name}
	}
}


%%
%% Tools for creating illustrations of memory ranges
%%

\newenvironment{memrange} {

	\newcommand{\alloc}[2]{
		\path (memrangeanchor) node[right=0ex, minimum width=##1, ##2] (currmemrangeanchor) {};
		\path (currmemrangeanchor.east) coordinate (memrangeanchor) {};
	}

	\newcommand{\remembersouth}[1]{
		\path (currmemrangeanchor.south) coordinate (##1) {}; }

	\newcommand{\remembersoutheast}[1]{
		\path (currmemrangeanchor.south east) coordinate (##1) {}; }

	\newcommand{\remembersouthwest}[1]{
		\path (currmemrangeanchor.south west) coordinate (##1) {}; }

	\begin{tikzpicture}[outer sep=0]
		% reset anchor used for appending allocations
		\path (0,0) coordinate (memrangeanchor) {};

} {
		% add some spacing to the right
		\path (memrangeanchor) node[right=0ex, minimum height=0, inner sep=0,
		                            minimum width=1ex] {};
	\end{tikzpicture}
}


% styles used for illustrating heap partitioning
\tikzstyle{heapalloc}  = [rectangle, minimum height=3ex, very thin, draw, draw opacity=0.5]
\tikzstyle{heapalloc1} = [heapalloc, top color=red!10!lightgray,
                                     bottom color=red!50!lightgray]
\tikzstyle{heapalloc2} = [heapalloc, top color=yellow!10!lightgray,
                                     bottom color=yellow!50!lightgray]
\tikzstyle{heapempty}  = [heapalloc, top color=white!10!lightgray,
                                     bottom color=white!50!lightgray]

% styles used for illustrating the component creation
\tikzstyle{memtype}     = [rectangle, minimum height=1ex, very thin, draw,
                           draw opacity=0.5]
\tikzstyle{memtext}     = [memtype, top color=red!10!lightgray,
                                    bottom color=red!50!lightgray]
\tikzstyle{memdata}     = [memtype, top color=yellow!10!lightgray,
                                    bottom color=yellow!50!lightgray]
\tikzstyle{memrodata}   = [memtype, top color=yellow!10!white,
                                    bottom color=yellow!50!white]
\tikzstyle{memavail}    = [memtype, top color=white!10!lightgray,
                                    bottom color=white!50!lightgray]
\tikzstyle{memmanaged}  = [memtype, top color=blue!10!lightgray,
                                    bottom color=blue!50!lightgray]
\tikzstyle{memreserved} = [memtype, top color=white!80!lightgray,
                                    bottom color=white!80!lightgray,
                                    pattern=north east lines,
                                    pattern color=black]

%%
%% Wheel used for illustrating the audio-out session interface
%%

\newenvironment{wheel}[1] {

	%%
	% Draw slices
	%
	% argument 1: range to iterate over the angles, in steps of 10
	% argument 2: style used for drawing the slices
	%
	\newcommand{\slices}[2]{
		\foreach \i in {##1} {
			\path[##2]
				(\i + 2:#1*0.2)  -- (\i + 2:#1*0.95) --
				(\i + 8:#1*0.95) -- (\i + 8:#1*0.2) --cycle;
		}
	}

	%%
	% Remember slice as tikz coordinate
	%
	% argument 1: angle
	% argument 2: name of coordinate
	% 
	\newcommand{\rememberouter}[2]{
		\path (##1 + 5:#1*0.95) coordinate (##2) {}; }

	\newcommand{\rememberinner}[2]{
		\path (##1 + 5:#1*0.85) coordinate (##2) {}; }

	\begin{tikzpicture}

		\path[draw, draw opacity=0.4, inner color=white, outer color=white!80!black, even odd rule]
			(0,0) circle (#1);

		\foreach \i in {0, 10,...,350} {
			\path[white, draw, draw opacity=0.5] (0,0) -- (\i:#1); }

} {
	\end{tikzpicture}
}


%%
%% API specification
%%

%
% Common color definitions for the API spec
%
\definecolor{apinamespace} {rgb}{1.0,0.95,0.78}
\definecolor{apimethod}    {rgb}{0.8,0.9,0.7}
\definecolor{apiclass}     {rgb}{0.8,0.85,0.9}
\definecolor{apifunction}  {rgb}{1.0,0.85,0.6}

%
% Tikz style definitions used in the API spec
%
\tikzstyle{apioutline}    = [draw=black!15]
\tikzstyle{apititlelabel} = [shape=signal, draw,
                             minimum height=3ex, top color=white,
                             draw opacity=0.3,
                             signal to=east]

\tikzstyle{apinamespacelabel} = [apititlelabel, bottom color=apinamespace]
\tikzstyle{apiclasslabel}     = [apititlelabel, bottom color=apiclass]
\tikzstyle{apimethodlabel}    = [apititlelabel, bottom color=apimethod]
\tikzstyle{apifunctionlabel}  = [apititlelabel, bottom color=apifunction]

\tikzstyle{apisectionlabel} = [at=(leftanchor), inner xsep=1ex, inner ysep=0.8ex,
                               text=apibg!20!black, font=\bfseries \small]


%%
% API description box
%
% argument 1: initial background color in r,g,b
%
\newenvironment{apibox}[1] {

	\path (0,0.5ex) node {}; % enforce a bit of spacing above the box

	\path (0,0)        coordinate (leftanchor);
	\path (leftanchor) coordinate (origleftanchor);

	\path (0.98\textwidth,0) coordinate (righttop);

	\definecolor{apibg} {rgb}{#1}

	\tikzstyle{apibg} = [fill=apibg!50!white]

	\newcommand{\apiboxtitle}[2]{

		% execute argument
		##1

		%
		% Define orig anchor (the background starts here), and anchor
		% (the position of the background to be processed)
		%
		\path (leftanchor |- namespacelabel.east)  coordinate (origleftanchor);
		\path (leftanchor |- namespacelabel.south) coordinate (leftanchor);

		%
		% Annotation at the right-top corner
		%
		\node[at=(righttop |- origleftanchor), anchor=north east,
		      text opacity=0.8, minimum size=2ex, inner sep=2ex] (annotation) {##2};

		% process first part of the background, including right counded corner
		\begin{scope}[on background layer]
			\path[apibg]               (origleftanchor)
			   {[rounded corners=5] -- (righttop |- origleftanchor)}
			                        -- (righttop |- leftanchor)
			                        -- (leftanchor) --cycle;
			\path[apioutline]          (origleftanchor)
			   {[rounded corners=5] -- (righttop |- origleftanchor)}
			                        -- (righttop |- leftanchor);
		\end{scope}

		% advance origleftanchor to be used as start pos for next bg step
		\path (leftanchor) coordinate (origleftanchor);
	}

	%%
	% Create API box title for a global namespace overview
	%
	% argument 1: annotation (at the top-right corner)
	%
	\newcommand{\apiglobalnamespaceboxtitle}[1]{

		\apiboxtitle{

			%
			% Invisible namespacelabel
			%
			\node[at=(leftanchor), anchor=north west]
				(namespacelabel) {};

		}{##1}
	}

	%%
	% Create API box title for a namespace overview
	%
	% argument 1: namespace
	% argument 2: annotation (at the top-right corner)
	%
	\newcommand{\apinamespaceboxtitle}[2]{

		\apiboxtitle{

			%
			% Segmented title label
			%
			\node[apinamespacelabel, at=(leftanchor), anchor=north west]
				(namespacelabel) {##1};

		}{##2}
	}

	%%
	% Create API box title for a global-function description
	%
	% argument 1: namespace
	% argument 2: function name
	% argument 3: annotation (at the top-right corner)
	%
	\newcommand{\apifunctionboxtitle}[3]{

		\apiboxtitle{

			%
			% Segmented title label
			%
			\node[apinamespacelabel, at=(leftanchor), anchor=north west, text opacity=0.5]
				(namespacelabel) {##1};
			\node[apifunctionlabel, at=(namespacelabel.east), anchor=west, signal from=west]
				(classlabel) {##2};

		}{##3}
	}

	%%
	% Create API box title for a class description
	%
	% argument 1: namespace
	% argument 2: class
	% argument 3: annotation (at the top-right corner)
	%
	\newcommand{\apiclassboxtitle}[3]{

		\apiboxtitle{

			%
			% Segmented title label
			%
			\node[apinamespacelabel, at=(leftanchor), anchor=north west, text opacity=0.5]
				(namespacelabel) {##1};
			\node[apiclasslabel, at=(namespacelabel.east), anchor=west, signal from=west]
				(classlabel) {##2};

		}{##3}
	}

	%%
	% Create API box title for a class method
	%
	% argument 1: namespace
	% argument 2: class
	% argument 3: function
	% argument 4: annotation (at the top-right corner)
	%
	\newcommand{\apimethodboxtitle}[4]{

		\apiboxtitle{

			%
			% Segmented title label
			%
			\node[apinamespacelabel, at=(leftanchor), anchor=north west, text opacity=0.5]
				(namespacelabel) {##1};
			\node[apiclasslabel, at=(namespacelabel.east), anchor=west, text opacity=0.5, signal from=west]
				(classlabel) {##2};
			\node[apimethodlabel, at=(classlabel.east), anchor=west, signal from=west]
				(functionlabel) {##3};
		}{##4}
	}

	%%
	% Append left-aligned content to API box
	%
	% argument: content
	%
	\newcommand{\apiboxcontent}[1]{

		\node[at=(leftanchor), anchor=north west, minimum size=1ex,
		      inner sep=1ex] (content) {##1};

		\path (leftanchor |- content.south) coordinate (leftanchor);
	}

	%%
	% Add vertical space to API box
	%
	% argument: space
	%
	\newcommand{\apiboxvspace}[1]{

		\path (leftanchor)+(0,-##1) coordinate (leftanchor);
	}

	%%
	% Append centered content to API box
	%
	% argument: content
	%
	\newcommand{\apiboxcenteredcontent}[1]{

		\path (leftanchor) -- coordinate (mid) (righttop |- leftanchor);
		\node[at=(mid), anchor=north, minimum size=1ex,
		      inner sep=1.5ex] (content) {##1};

		\path (leftanchor |- content.south) coordinate (leftanchor);
	}

	%%
	% Append section to API box
	%
	% argument 1: section label text
	% argument 2: section background color as r,g,b
	%
	\newcommand{\apisection}[2]{

		% create section label with transparent text to obtain its bounding box
		\node[apisectionlabel, at=(leftanchor), text opacity=0, anchor=north west]
			(sectionlabel) {##1};

		% advance anchor to extend the old background to the south of the tab
		\path (sectionlabel.south west) coordinate (leftanchor);

		%
		% Complete the background of the previous part of the API box
		%
		\begin{scope}[on background layer]
			\path[apibg] (origleftanchor)
			          -- (righttop |- origleftanchor)
			          -- (righttop |- leftanchor)
			          -- (leftanchor)
			          --cycle;

		\end{scope}

		% switch to using the new color
		\definecolor{apibg} {rgb}{##2}

		% helper coordinates for drawing the tab
		\path (sectionlabel.south east)+(4ex,1ex) coordinate (mid);
		\path (sectionlabel.south west)+(0,-2ex)  coordinate (below);

		%
		% Draw tab hosting the section label
		%
		\path [top color=apibg!90!black, bottom color=apibg!50!white]
		      (sectionlabel.north west)
		         {[rounded corners=7] -- (sectionlabel.north east)
		                              -- (mid)}
		                              -- (righttop |- mid)
		                              -- (righttop |- below)
		                              -- (sectionlabel.south west |- below)
		                              --cycle;

		\path[apioutline] (origleftanchor)
		               -- (leftanchor);
		\path[apioutline] (righttop |- origleftanchor)
		               -- (righttop |- leftanchor);

		\path [draw=apibg!80!black]
		      (sectionlabel.north west)
		         {[rounded corners=7] -- (sectionlabel.north east)
		                              -- (mid)
		                              -- (righttop |- mid)};

		%
		% Another instance of the section label with visible text
		%
		\node[apisectionlabel, at=(sectionlabel.north west), anchor=north west] {##1};

		\path (leftanchor)                       coordinate (origleftanchor);
		\path (leftanchor |- sectionlabel.south) coordinate (leftanchor);
	}
} {
	% make sure the box is large enough to fit the annotation
	\node[fit=(leftanchor) (annotation.south)] (maxhelper) {};
	\path (leftanchor |- maxhelper.south) coordinate (leftanchor);

	%
	% Finalize background up to the current leftanchor
	%
	\begin{scope}[on background layer]
		\path[apibg]               (origleftanchor)
		                        -- (righttop |- origleftanchor)
		  {[rounded corners=5]  -- (righttop |- leftanchor)
		                        -- (leftanchor)} --cycle;
	\end{scope}

	\path[apioutline]          (righttop |- origleftanchor)
	  {[rounded corners=5]  -- (righttop |- leftanchor)
	                        -- (leftanchor)
	                        -- (origleftanchor)};

	\path (leftanchor)+(0,-0.5ex) node {}; % enforce a bit of spacing below the box
}
